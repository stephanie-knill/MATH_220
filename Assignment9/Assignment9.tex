% ======================= Pre-Amble =========================
      
%Format
\documentclass[11pt, oneside]{article}   	% use "amsart" instead of "article" for AMSLaTeX format 
                     						%imports package {article} and specify option(s) [11pt, oneside]
\usepackage{geometry}                		% See geometry.pdf to learn the layout options. There are lots. 
    \geometry{letterpaper}                   		% ... or a4paper or a5paper or ... 
    %\geometry{landscape}                		% Activate for rotated page geometry

\usepackage[parfill]{parskip}    		        % Activate to begin paragraphs with an empty line rather than an indent

    %Colours
    \usepackage{graphicx, subcaption}
    \usepackage[usenames, dvipsnames]{color}     % font colour:    \textcolor{<colour>}{text}
          									%highlight text:  \colorbox{<color>}{text}
    \usepackage{soul}						%highlight text: \hl{}     %only  yellow								
    									%list of colours: https://www.sharelatex.com/learn/Using_colours_in_LaTeX
    									
    %Bullets
    \usepackage{enumerate}     %specify type of enumeration: \being{enumerate}[<type of enumeration>]
    
    %Footnote Spacing
    \setlength{\footnotesep}{0.4cm}                  %specify spacing b/w footnotes
    \setlength{\skip\footins}{0.6cm}                    % space b/w footnotes and textbody


%Mattematics
    %American Mathematics Society packages
    \usepackage{amsmath, amsfonts}	   %math, fonts?
    \usepackage{amssymb}       %symbols
    \usepackage{amsthm}          %theorems
    \newtheorem{proposition}{Proposition}

    %QED
    \newcommand*{\QEDA}{\hfill\ensuremath{\blacksquare}}         %make qed filled square:    \QEDA
    \newcommand*{\QEDB}{\hfill\ensuremath{\square}}               %make qed empty square: \QEDB 
    
    \renewcommand\qedsymbol{\ensuremath{\blacksquare}}		%Proof environment


%Figures
\usepackage{caption}
\captionsetup[figure]{labelfont=bf}    %make figure labels boldface
\captionsetup[table]{labelfont=bf}     %make table labels boldface

\usepackage[hidelinks]{hyperref}                % Allows for clickable references

    %Tables
    \usepackage[none]{hyphenat}                    % Stops breaking-up words in a table (i.e. no hyphens)                                                             
    
    \usepackage{array}   
    \newcolumntype{x}[1]{>{\centering\let\newline\\\arraybackslash\hspace{0pt}}p{#1}}       %center fixed column width: x{<len>}                      
    \newcolumntype{$}{>{\global\let\currentrowstyle\relax}}                                                   % let us apply things (e.g. bold/italicize) to entire row            
    \newcolumntype{^}{>{\currentrowstyle}}
    \newcommand{\rowstyle}[1]{\gdef\currentrowstyle{#1} #1\ignorespaces}
    
    %Images
    \graphicspath{ {images/} }                          %directory that your images are located in within your current directory
    
    %Diagrams
    \usepackage[latin1]{inputenc}
    \usepackage{tikz}
    \usepackage{tkz-berge}
    \usetikzlibrary{shapes,arrows}


%Bibliography
\usepackage[numbers,sort&compress]{natbib}   %for multiple references: sorts  (i.e. [1,2] NOT [2, 1] )
                                           				  %                                     compresses (i.e. [1-3] )
\usepackage[nottoc]{tocbibind}                            %add bibliography to table of contents


%Miscellaneous
\usepackage{dirtytalk}    %quotations: use \say  
\newcommand{\N}{\mathbb N}
\newcommand{\Z}{\mathbb Z}
\newcommand{\R}{\mathbb R}
\newcommand{\Q}{\mathbb Q}


%=========== Header & Footer =========================
\usepackage{fancyhdr}
\usepackage{lastpage}      %ensures you can reference LastPage (i.e. Page 2 of 10)

\renewcommand{\headrulewidth}{0.4pt}		%Decorative Header line: thickness={0.4pt}
\renewcommand{\footrulewidth}{0.4pt}		%Decorative Footer line: thickness={0.4pt}

\setlength{\headheight}{13.6pt} 		%space b/w top of page & header
\setlength{\headsep}{0.3in}		%space b/w page header and body

%Make Header & Footer    
\pagestyle{fancy}
    \lhead{Stephanie Knill} 		% controls the left corner of the header
    \chead{} 					% controls the center of the header
    \rhead{} 					% controls the right corner of the header
    \lfoot{} 					% controls the left corner of the footer
    \cfoot{Page~\thepage\ of \pageref{LastPage}} 				% controls the center of the footer
    												%Page~\thepage\  if just want Page x
    \rfoot{}			 		% controls the right corner of the footer

% ======================== Document ======================
\begin{document}

% Title Page
\title{MATH 220 --- Assignment 9 \\
\line(1,0){360} \\              %(slope x, y){length of line}
}
\author{
Stephanie Knill \\
54882113 \\
Due: April 5, 2016}

\date{}                   % Activate:  display a given date (e.g. {August 4} ) or no date (empty {} )
                                    %No activate: display current date
\maketitle

%\thispagestyle{empty}                   %Remove header from this (first) page. Change empty -> plain to keep numbering
%								-> Doesn't matter in this case (b/c title page)
%\cleardoublepage


% ================= Questions ================

\section*{Question 1}

\begin{proposition}
There \textit{does not} exist a continuous bijective function $f: \mathbb{R} \mapsto \mathbb{R} - \{ 1\}$.
\end{proposition}

\begin{proof}
Assume, to the contrary, that there exists a continuous bijective function $f: \mathbb{R} \mapsto \mathbb{R} - \{ 1\}$. Since $f$ is bijective, then $f$ is also surjective. So there exists $x_1, x_2 \in \mathbb{R}$ such that $f(x_1)=0$ and $f(x_2)=2$. Since $f$ is continuous and $0<1<2$, then by the Intermediate Value Theorem, there exists an $x_3 \in \mathbb{R}$ such that $f(x_3)=1$. However, $1 \notin \mathbb{R} - \{1\}$, thereby giving us the necessary contradiction.
\end{proof}


\section*{Question 2}

Let $A_1, A_2, B_1, B_2$ be non-empty sets such that $|A_i|=|B_i|$ for $i=1,2$. Then
\begin{enumerate}[\quad(a)]
	\item $|A_1 \times A_2| = |B_1 \times B_2|$
	\begin{proof} 
		\emph{Case 1:} $A_1, A_2, B_1, B_2$ finite.
		Let $|A_1| =  |A_2| = |B_1| = |B_2| = m$, for some $m \in \mathbb{N} \cup \{0\}$. Then $|A_1 \times A_2| = m^2 = |B_1 \times B_2|$.
		
		\emph{Case 2:} $A_1, A_2, B_1, B_2$ infinite.
		Thus $|A_1| =  |A_1 \times A_2|$ and $|B_1| = |B_1 \times B_2|$. Since $|A_1| =  |B_1|$, then $|A_1 \times A_2| = |B_1 \times B_2|$.
	\end{proof}
	\item If $A_1 \cap A_2 = B_1 \cap B_2 = \emptyset$, Then $|A_1 \cup A_2| = |B_1 \cup B_2|$.
	\begin{proof} 
		\emph{Case 1:} $A_1, A_2, B_1, B_2$ finite.
		Let $|A_1| =  |A_2| = |B_1| = |B_2|=m$, for some $m \in \mathbb{N} \cup \{0\}$. Since the intersection is empty, then $|A_1 \cup A_2| = 2m= |B_1 \cup B_2|$.	
			
		\emph{Case 2:} $A_1, A_2, B_1, B_2$ infinite.
		Thus $|A_1| =  |A_1 \cup A_2|$ and $|B_1| = |B_1 \cup B_2|$. Since $|A_1| =  |B_1|$, then $|A_1 \cup A_2| = |B_1 \cup B_2|$.

	\end{proof}
	\end{enumerate}


\section*{Question 3}

\begin{proposition}
Let $A$ be an non-empty set. Prove that $|A| \leq |A \times A|$
\end{proposition}

\begin{proof} 
	\emph{Case 1:} $A$ finite. Since $A\neq \emptyset$, then $|A| < |A \times A|$.
		
	\emph{Case 2:} $A$ infinite. By our in class Theorem, $|A| = |A \times A|$. Thus for any set $A$, we have that $|A| \leq |A \times A|$.
\end{proof} 


\section*{Question 4}

\begin{proposition}
If $A$ is a denumerable set and there exists a surjective function from $A$ to $B$ (and $B$ is infinite), then $B$ is denumerable.
\end{proposition}

\begin{proof} 
If $B$ is an infinite set, then it is either denumerable or uncountable. Let us assume to the contrary that $B$ is denumerable. Since there exists a surjective function $f : A \mapsto B$, then for all $b \in B$ there exists an $a \in A$ such that $f(a)=b$. Since $B$ is uncountable, we know that $A$ is also uncountable. However, we know that $A$ is denumerable, thus giving us the necessary contradiciton.
\end{proof} 

\section*{Question 5}

\begin{proposition}
If $A$ and $B$ are denumerable sets, and $C$ is a finite set, then $A \cup B \cup C$ is denumerable.
\end{proposition}

\begin{proof} 
Since $A$ and $B$ are denumerable, then $A \cup B$ is denumerable. Thus the union of a denumerable set $(A \cup B)$ with a finite set $C$ is also denumerable. That is, $A \cup B \cup C$ is denumerable.
\end{proof} 



\section*{Question 6}

\begin{proposition}
If a set $A$ contains an uncountable subset, then $A$ is uncountable
\end{proposition}

\begin{proof} 
Let $B \subseteq A$, where $B$ is uncountable. Assume to the contrary that $A$ is countable. Then by our in class Theorem, we know that any subset of a countable set is countable. However, $B \subseteq A$ and $B$ is uncountable, thereby giving us the necessary contradiction.
\end{proof} 


\section*{Question 7}

\begin{proposition}
Let $A$ be any uncountable set, and let $B \subset A$ be a countable subset of $A$. Prove that $|A| = |A - B|$.
\end{proposition}

\begin{proof} 
Assume that $A$ is any uncountable set and $B \subset A$ be a countable subset of $A$. So $A-B$ is infinite. Let us define a new denumerable subset $C \subseteq A-B$. From our in class proof, we know that $B \cup C$ is also denumerable. Thus there exists a bijective function $f : B \cup C \mapsto C$. Now, let us define a function $g : A \mapsto A - B$ such that
	\begin{align*}
		f(x) = \begin{cases}
				f(x) & \text{if } x \in B \cup C \\
				x & \text{if } x \in (A-B)-C
			\end{cases}
	\end{align*}
which is bijective.
\end{proof} 


\section*{Question 8}

\begin{enumerate}[\quad(a)]

	\item If $\mathcal{P}_{fin}(\mathbb{N})$ denotes the set of finite subsets of $N$, show that $\mathcal{P}_{fin}(\mathbb{N})$ is denumerable.
	\begin{proof} 
		Since $\mathcal{P}_{fin}(\mathbb{N}) \subset \mathbb{N}$ and $\mathbb{N}$ is denumerable, we have that
		$$|\mathcal{P}_{fin}(\mathbb{N})| = |\mathbb{N}|$$
		Thus $\mathcal{P}_{fin}(\mathbb{N})$ is also denumerable.
	\end{proof} 
	
	\item If $\mathcal{P}_{inf}(\mathbb{N})$ denotes the set of finite subsets of $N$, show that $\mathcal{P}_{inf}(\mathbb{N})$ is uncountable.
	\begin{proof} 
		Since $\mathcal{P}_{fin}(\mathbb{N}) \subset \mathcal{P}(\mathbb{N})$ and $\mathcal{P}(\mathbb{N})$ is uncountable, we have that
		$$|\mathcal{P}_{inf}(\mathbb{N})| = |\mathcal{P}(\mathbb{N})|$$
		Thus $\mathcal{P}_{inf}(\mathbb{N})$ is also uncountable.
	\end{proof} 

\end{enumerate}

\section*{Question 9}
\begin{proposition}
Let $A, B$ be sets. If $|A - B| = |B - A|$ then $|A| = |B|$.
\end{proposition}

\begin{proof} 
	\emph{Case 1:} $A \cap B = \emptyset$.
		
		Then $A-B=A$ and $B-A=B$, so we have that $|A-B|=|A|$ and $|B-A|=|B|$. Therefore $|A-B|= |B-A|= |A|=|B|$.
		
	\emph{Case 2:} $A\cap B \neq \emptyset$
	
	Since $A=(A-B) \cup (A \cap B)$ and $B=(B-A) \cup (A \cap B)$, then we have that
	$$|A|=|(A-B) \cup (A \cap B)|, \text{ and}$$
	$$|B|=|(B-A) \cup (A \cap B)|$$
	
	However, since $|A - B| = |B - A|$, then
	$|A| = |(A-B) \cup (A \cap B)| = |(B-A) \cup (A \cap B)| =|B|$
\end{proof} 

\section*{Question 10}
\begin{proposition}
Let $\{0, 1\}^{\mathbb{N}}$ be the set of all possible sequences of 0s and 1s. We have proved in class that this set is uncountable. Corollary 10.22 in the text states that in fact, the cardinality of this set is continuum: $|\mathbb{R}| = |\{0, 1\}^{\mathbb{N}}|$. Using this fact, prove that $|\mathbb{R} � \mathbb{R}| = |\mathbb{R}|$.
\end{proposition}

\begin{proof} 

\end{proof} 



















\end{document} 