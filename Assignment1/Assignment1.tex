% ======================= Pre-Amble =========================

\documentclass[11pt, oneside]{article}   	% use "amsart" instead of "article" for AMSLaTeX format 
                     						%imports package {article} and specify option(s) [11pt, oneside]
\usepackage{geometry}                		% See geometry.pdf to learn the layout options. There are lots.                                        

\geometry{letterpaper}                   		% ... or a4paper or a5paper or ... 
%\geometry{landscape}                		% Activate for rotated page geometry

\usepackage[parfill]{parskip}    		        % Activate to begin paragraphs with an empty line rather than an indent

\usepackage[hidelinks]{hyperref}                % Allows for clickable references

%American Mathematics Society packages
\usepackage{amsmath}	   %math
\usepackage{amssymb}       %symbols
\usepackage{amsthm}          %theorems

%Graphics
\usepackage{graphicx}
\usepackage[usenames, dvipsnames]{color}     % font colour:    \textcolor{<colour>}{text}
      									%highlight text:  \colorbox{<color>}{text}
									
									%list of colours: https://www.sharelatex.com/learn/Using_colours_in_LaTeX

%Images		                
\graphicspath{ {images/} }                          %directory that your images are located in within your current directory
	

%Footnote Spacing
\setlength{\footnotesep}{0.4cm}                  %specify spacing b/w footnotes
\setlength{\skip\footins}{0.6cm}                    % space b/w footnotes and textbody

%Table
\usepackage[none]{hyphenat}                    % Stops breaking-up words in a table (i.e. no hyphens)
                                                               

\usepackage{array}   
\newcolumntype{x}[1]{>{\centering\let\newline\\\arraybackslash\hspace{0pt}}p{#1}}       %center fixed column width: x{<len>}                      
\newcolumntype{$}{>{\global\let\currentrowstyle\relax}}                                                   % let us apply things (e.g. bold/italicize) to entire row            
\newcolumntype{^}{>{\currentrowstyle}}
\newcommand{\rowstyle}[1]{\gdef\currentrowstyle{#1} #1\ignorespaces}

%Bibliography
\usepackage[numbers,sort&compress]{natbib}   %for multiple references: sorts  (i.e. [1,2] NOT [2, 1] )
                                           				  %                                     compresses (i.e. [1-3] )
\usepackage[nottoc]{tocbibind}                            %add bibliography to table of contents

%Diagrams
\usepackage[latin1]{inputenc}
\usepackage{tikz}
\usetikzlibrary{shapes,arrows}
	

\usepackage{dirtytalk}    %quotations: use \say  

\usepackage{caption}
\captionsetup[figure]{labelfont=bf}    %make figure labels boldface
\captionsetup[table]{labelfont=bf}     %make table labels boldface

%Bullets
\usepackage{enumerate}     %specify type of enumeration: \being{enumerate}[<type of enumeration>]

%QED
\newcommand*{\QEDA}{\hfill\ensuremath{\blacksquare}}         %make qed filled square:    \QEDA
%\newcommand*{\QEDB}{\hfill\ensuremath{\square}}               %make qed empty square: \QEDB 

%Header and Footer
\usepackage{fancyhdr}
\usepackage{lastpage}      %ensures you can reference LastPage (i.e. Page 2 of 10)


%=========== Header & Footer =========================

\pagestyle{fancy}
\lhead{Stephanie Knill} 		% controls the left corner of the header
\chead{} 					% controls the center of the header
\rhead{} 					% controls the right corner of the header
\lfoot{} 					% controls the left corner of the footer
\cfoot{Page~\thepage\ of \pageref{LastPage}} 				% controls the center of the footer
												%Page~\thepage\  if just want Page x
\rfoot{}			 		% controls the right corner of the footer
\renewcommand{\headrulewidth}{0.4pt}
\renewcommand{\footrulewidth}{0.4pt}

% ======================== Document ======================
\begin{document}


\title{MATH 220 --- Assignment 1 \\
\line(1,0){360} \\              %(slope x, y){length of line}
\vspace{0.2cm}}
\author{
Stephanie Knill \\
54882113 \\
Due: January 12, 2015}

\date{}                   % Activate:  display a given date (e.g. {August 4} ) or no date (empty {} )
                                    %No activate: display current date
\maketitle

\thispagestyle{empty}                   %Remove header from this (first) page. Change empty -> plain to keep numbering


% ================= Questions ================

\section*{Question 1}

Let $S = \{-2, -1, 0, 1, 2, 3\}$. Then we can describe the sets $A, B, C, D$ as

\begin{enumerate}[ (a)]           
    \item $A = \{1, 2, 3\} = \{\,x \in S \mid x \geq 1 \,\} = \{\,x \in S \mid \text{$x$ is positive} \,\}$       %\, : thin space
    \item $B = \{0, 1, 2, 3\} = \{\,x \in S \mid x \geq 0 \,\} = \{\,x \in S \mid \text{$x$ is nonnegative} \,\}$    
    \item $C = \{-2,-1\} = \{\,x \in S \mid x \leq -1 \,\} = \{\,x \in S \mid x < 0 \,\} = \{\,x \in S \mid \text{$x$ is negative} \,\}$   
    \item $D = \{-2,2,3\} = \{\,x \in S \, : \, |x| \geq 2 \,\} = \{\, x \in S \, : \, x \geq 2$ $and$ $x \leq -2 \, \}$
        
\end{enumerate}


\section*{Question 2}

\begin{enumerate}[ (a)]           
    \item $A = \{\,n \in \mathbb{Z} \mid -4 < n \leq 4 \,\} = \{-4, -3, -2, -1, 0, 1, 2, 3, 4 \} = \{-4, -3, -2, \ldots , 4\}$
    \item $A = \{\,n \in \mathbb{Z} \mid n^2 < 5 \,\} = \{-2, -1, 0, 1, 2\}$
    \item $A = \{\,n \in \mathbb{Z} \mid n^3 < 100 \,\} = \{-4, -3, -2, -1, 0, 1, 2, 3, 4 \} = \{-4, -3, -2, \ldots , 4 \}$
    \item $A = \{\,x \in \mathbb{R} \mid x^2-x = 0 \,\} = \{0, 1\}$
    \item $A = \{\,x \in \mathbb{R} \mid x^2 + 1 = 0 \,\} = \{ \, \} = \emptyset$
        
\end{enumerate}

\section*{Question 3}

\begin{enumerate}[ (a)]           
    \item The sets $A = \{1, 2\}, B = \{1, 2\}, C = \{1, 2, 3, 4\}$ have the property $A \subseteq B \subset C$.
    \item Let $A = \{1, 2\}, B = \{ \, \{1, 2\}, 3 \}, C = \{ \, \{\{1, 2\}, 3 \}, 4, 5\}$.
    Then $A \in B, \, B \in C,$ and $A \notin C$.
    \item Let $A = \emptyset, B = \{\emptyset, 1, 2\}, C = \{-1, 0, 100\}$. Then $A \in B$ and $A \subset C$
        
\end{enumerate}

\section*{Question 4}

For the set $A = \{0, \emptyset, \{\emptyset \}\}$, we have $$\mathcal{P}(A) = \{\emptyset, \{0\}, \{\emptyset\}, \{\{\emptyset \}\}, \{0, \emptyset\}, \{0, \{\emptyset \}\}, \{\emptyset, \{\emptyset\}\}, \{0, \emptyset, \{\emptyset \}\} \,  \}$$ and $|\mathcal{P}(A)| = 2^{|A|} = 2^3 = 8$

\section*{Question 5}

\begin{enumerate}[ (a)]
           
    \item Conjecture: If $\{1\} \in \mathcal{P}(A)$, then $1 \in A$.
    \vspace{0.3cm}\\
    \textbf{Proof}
    \vspace{0.1cm}\\
    By definition, the power set of $\mathcal{P}(A)$ is the set consisting of all subsets of $A$. If $\{1\} \in \mathcal{P}(A)$, then $1$ is an element in $A$, hence $1 \in A$. \QEDA
    
    \item Conjecture: If $\{1\} \in \mathcal{P}(A)$, then $1 \notin A$.
    \vspace{0.3cm}\\
    \textbf{False}. Let $\mathcal{P}(A) = \{ \emptyset, \{1\}, \{2\}, \{1, 2\} \}$, then $A = \{1, 2\}$, Since $1 \in A$ is true, the statement $1 \notin A$ is false. Therefore the conjecture is also false.
    
    \item Conjecture:     If four sets $A, B, C, D$ are subsets of $\{1, 2, 3\}$ such that $|A| = |B| = |C| = |D| = 2$, then at least two of these sets are equal.
    \vspace{0.3cm}\\
    \textbf{Proof}
    \vspace{0.1cm}\\
    The power set of \{1, 2, 3\} is given by 
    $$ \{1, 2, 3\} = \{ \emptyset, \{1\}, \{2\}, \{3\}, \{1, 2\}, \{1, 3\}, \{2, 3\}, \{1, 2, 3\} \}$$
   Since $|A| = |B| = |C| = |D| = 2$, we are only interested in the 2-elements sets in the power series. In this case, there are 3 sets $\{1, 2\}, \{1, 3\},$ and $\{2, 3\}$. Since there are only 3 possible subsets to be assigned to the 4 sets $A, B, C,$ and $D$, then either all the sets are equal, three sets are equal or two sets are equal. Thus we can conclude that at least two of the sets must be equal. \QEDA

    \item Conjecture: $A \subset \mathcal{P}(B)$ and $|A| = 2$, then $B$ has at least two elements.
    \vspace{0.3cm}\\
    \textbf{Proof}
    \vspace{0.1cm}\\
    Since $A$ is a proper subset of $\mathcal{P}(B)$ that contains 2 elements, then $|\mathcal{P}(B)| \geq 4$. Using the definition of the power set, we have that $|\mathcal{P}(B)| = 2^{|B|} \geq 4$, or $|B| \geq 2$. Thus we can conclude that the set $B$ has at least two elements. \QEDA
        
\end{enumerate}

\end{document} 