% ======================= Pre-Amble =========================
      
%Format
\documentclass[11pt, oneside]{article}   	% use "amsart" instead of "article" for AMSLaTeX format 
                     						%imports package {article} and specify option(s) [11pt, oneside]
\usepackage{geometry}                		% See geometry.pdf to learn the layout options. There are lots. 
    \geometry{letterpaper}                   		% ... or a4paper or a5paper or ... 
    %\geometry{landscape}                		% Activate for rotated page geometry

\usepackage[parfill]{parskip}    		        % Activate to begin paragraphs with an empty line rather than an indent

    %Colours
    \usepackage{graphicx, subcaption}
    \usepackage[usenames, dvipsnames]{color}     % font colour:    \textcolor{<colour>}{text}
          									%highlight text:  \colorbox{<color>}{text}
    \usepackage{soul}						%highlight text: \hl{}     %only  yellow								
    									%list of colours: https://www.sharelatex.com/learn/Using_colours_in_LaTeX
    									
    %Bullets
    \usepackage{enumerate}     %specify type of enumeration: \being{enumerate}[<type of enumeration>]
    
    %Footnote Spacing
    \setlength{\footnotesep}{0.4cm}                  %specify spacing b/w footnotes
    \setlength{\skip\footins}{0.6cm}                    % space b/w footnotes and textbody


%Mattematics
    %American Mathematics Society packages
    \usepackage{amsmath}	   %math
    \usepackage{amssymb}       %symbols
    \usepackage{amsthm}          %theorems

    %QED
    \newcommand*{\QEDA}{\hfill\ensuremath{\blacksquare}}         %make qed filled square:    \QEDA
    %\newcommand*{\QEDB}{\hfill\ensuremath{\square}}               %make qed empty square: \QEDB 


%Figures
\usepackage{caption}
\captionsetup[figure]{labelfont=bf}    %make figure labels boldface
\captionsetup[table]{labelfont=bf}     %make table labels boldface

\usepackage[hidelinks]{hyperref}                % Allows for clickable references

    %Tables
    \usepackage[none]{hyphenat}                    % Stops breaking-up words in a table (i.e. no hyphens)                                                             
    
    \usepackage{array}   
    \newcolumntype{x}[1]{>{\centering\let\newline\\\arraybackslash\hspace{0pt}}p{#1}}       %center fixed column width: x{<len>}                      
    \newcolumntype{$}{>{\global\let\currentrowstyle\relax}}                                                   % let us apply things (e.g. bold/italicize) to entire row            
    \newcolumntype{^}{>{\currentrowstyle}}
    \newcommand{\rowstyle}[1]{\gdef\currentrowstyle{#1} #1\ignorespaces}
    
    %Images
    \graphicspath{ {images/} }                          %directory that your images are located in within your current directory
    
    %Diagrams
    \usepackage[latin1]{inputenc}
    \usepackage{tikz}
    \usepackage{tkz-berge}
    \usetikzlibrary{shapes,arrows}


%Bibliography
\usepackage[numbers,sort&compress]{natbib}   %for multiple references: sorts  (i.e. [1,2] NOT [2, 1] )
                                           				  %                                     compresses (i.e. [1-3] )
\usepackage[nottoc]{tocbibind}                            %add bibliography to table of contents


%Miscellaneous
\usepackage{dirtytalk}    %quotations: use \say  



%=========== Header & Footer =========================
\usepackage{fancyhdr}
\usepackage{lastpage}      %ensures you can reference LastPage (i.e. Page 2 of 10)

\renewcommand{\headrulewidth}{0.4pt}		%Decorative Header line: thickness={0.4pt}
\renewcommand{\footrulewidth}{0.4pt}		%Decorative Footer line: thickness={0.4pt}

\setlength{\headheight}{13.6pt} 		%space b/w top of page & header
\setlength{\headsep}{0.3in}		%space b/w page header and body

%Make Header & Footer    
\pagestyle{fancy}
    \lhead{Stephanie Knill} 		% controls the left corner of the header
    \chead{} 					% controls the center of the header
    \rhead{} 					% controls the right corner of the header
    \lfoot{} 					% controls the left corner of the footer
    \cfoot{Page~\thepage\ of \pageref{LastPage}} 				% controls the center of the footer
    												%Page~\thepage\  if just want Page x
    \rfoot{}			 		% controls the right corner of the footer

% ======================== Document ======================
\begin{document}

% Title Page
\title{MATH 220 --- Assignment 4 \\
\line(1,0){360} \\              %(slope x, y){length of line}
}
\author{
Stephanie Knill \\
54882113 \\
Due: February 2, 2016}

\date{}                   % Activate:  display a given date (e.g. {August 4} ) or no date (empty {} )
                                    %No activate: display current date
\maketitle

%\thispagestyle{empty}                   %Remove header from this (first) page. Change empty -> plain to keep numbering
%								-> Doesn't matter in this case (b/c title page)
%\cleardoublepage


% ================= Questions ================

\section*{Question 1}

\begin{enumerate}[ (a)]           
    \item For ever rational number $r$, the number $1/r$ is rational.
    
    $P: \; \forall r \in \mathbb{Q}, \; 1/r$ is rational.
    
    $\sim P: \; \exists r \in \mathbb{Q} \; :  \; 1/r$ is irrational. 
    
    \item There exists a rational number $r$ such that $r^2=2$.
    
    $P: \; \exists r \in \mathbb{Q} \; : \; r^2 =2.$
    
    $\sim P: \; \forall r \in \mathbb{Q}, \; r^2 \neq 2.$
\end{enumerate}


\section*{Question 2}

Let $P$ be the statement \say{$\forall x \in \mathbb{R} \; \exists y \in \mathbb{R}, \; y^2 = x$}.

\begin{enumerate}[ (a)]           
    \item In words: \say{For all real numbers $x$, there exists a real number $y$ such that $y^2=x$}   
    \item This statement $P$ is \textbf{True}.
    \item $\sim P: \; \exists x \in \mathbb{R} \; \forall{y} \in \mathbb{R}, \; y^2 \neq x$
        
\end{enumerate}

\section*{Question 3}

\begin{enumerate}[ (a)]    

	\item $\exists x \in \mathbb{R}, x^2-x=0$
	
	\textbf{True:} $x=0, 1$
	
	\item $\forall x \in \mathbb{R}, \sqrt{x^2} = x$
	
	\textbf{True}
	
	\item $\exists x \in \mathbb{R}, \exists y \in \mathbb{R}, x+y+3=8$
	
	\textbf{True:} $x=0$ and  $y=5$.
	
	\item $\forall x,y \in \mathbb{R}, x+y+3=8$
	
	\textbf{False:} $x = y = 0$.
	
	\item$\forall x \in \mathbb{R}, \exists y \in \mathbb{R}, x+y+3=8$
	
	\textbf{True:} $y = 5-x$
	
	\item $\exists x \in \mathbb{R}, \forall y \in \mathbb{R}, x+y+3 =8$
	
	\textbf{True:} $x = 5-y$
	
	\item $\exists m,n \in \mathbb{N}, n^2+m^2 =25$
	
	\textbf{True:} $m=3, n=4$
	
	\item $\forall m \in \mathbb{N}, \exists n \in \mathbb{N}, n^2+m^2=25$
	
	\textbf{False:} $m=100$

\end{enumerate}

\section*{Question 4}

Let $P(x)$ and $Q(x)$ be open sentences where the domain of the variable $x$ is a set $S$. Then which of the following implies $\sim P(x) \Rightarrow Q(x)$ is false for some $x \in S$ ?

If $P(x)$ is False and $Q(x)$ is False, then $\sim P(x) \Rightarrow Q(x)$ is False. Thus, for the following:

\begin{enumerate}[ (a)]    

	\item $P(x) \land Q(x)$ is false for all $x \in S$: \textbf{does imply False}.
	
	Here, we have 3 cases for the truth values of $P(x)$ and $Q(x)$: 
	\begin{enumerate}[\qquad 1.]
		\item Both $P(x)$ and $Q(x)$ are False
		\item $P(x)$ is True and $Q(x)$ is False
		\item $P(x)$ is False and $Q(x)$ is True
	\end{enumerate}	
	
	Since the first case would make $\sim P(x) \Rightarrow Q(x)$ False, then  \say{$P(x) \land Q(x)$ is false for all $x \in S$} implies that $\sim P(x) \Rightarrow Q(x)$ is False for some $x \in S$.
	
	\item $P(x)$ is true for all $x \in S$: \textbf{does not imply False.}
	
	Since $P(x)$ always True, then $\sim P(x) \Rightarrow Q(x)$ can never be False.
	
	\item $Q(x)$ is true for all $x \in S$: \textbf{does not imply False.}
	
	Since $Q(x)$ always True, then $\sim P(x) \Rightarrow Q(x)$ can never be False.
	
	\item $P(x) \lor Q(x)$ is false for all $x \in S$: \textbf{does imply False.}
	
	Here both $P(x)$ and $Q(x)$ are False, therefore implying that $\sim P(x) \Rightarrow Q(x)$ is False.
\end{enumerate}


\section*{Question 5}

Let $S=[1,2]$ and $T=(3, \infty)$.

\begin{enumerate}[ (a)]    
	\item $\exists x \in S$ s.t. $\exists y \in T$ s.t. $|x-y|> 3$: \say{There exists an $x$ in the set $S$ such that there exists a $y$ in the set $T$ such that $|x-y| > 3$.}
	
	\textbf{Proof:} Let $x=1$ and $y=100$. Since the inequality
	$$3 < |1-100| = |-99| = 99$$
	holds true, then statement is also true. \QEDA
	
	\item $\exists x \in S$ s.t. $\forall y \in T, |x-y| > 3$: \say{There exists an $x$ in the set $S$ such that for all $y$ in the set $T$, $|x-y| > 3$.}
	
	\textbf{Proof:} Let $x=1$. Then the inequality can be expressed as
	$$3 < |1-y| = 1+ |-y| $$
	and we have that $|y| > 2$. Since $y \in T$, then $y > 3$. Therefore the statement holds true. \QEDA
	
	\item $\forall x \in S, \exists y \in T$, s.t. $|x-y| > 3$: \say{For all $x$ in the set $S$, there exists a $y$ in the set $T$ such that $|x-y| > 3$.}
	
	\textbf{Proof:} Let $y=100$. Then the inequality can be expressed as
	$$3 < |x-100| = |x| + 100 $$
	and we have that $|x| > -97$. Since $x \in T=[1,2]$, then the inequality holds true for all $x$. \QEDA
	
	\item $\forall x \in S, \forall y \in T, |x-y| > 3$: \say{For all $x$ in the set $S$ and for all $y$ in the set $T$, $|x-y| > 3$.}
	
	\textbf{Proof:} We can express the inequality as
	\begin{align*}
		|x-y| & > 3 \\
		|x| + |-y| & > 3 \\
		|x| + |y| & > 3 \\
	\end{align*}
	Since $x \in S =[1,2]$ and $y \in T=(3, \infty)$, then $|x| \geq 1$ and $|y| > 3$. Thus
	$$|x| + |y| > 1+3 > 3$$
	and the inequality holds true for all $x$ and $y$. \QEDA

\end{enumerate}

\section*{Question 6}

Let $I=\{n^2 | n \in \mathbb{Z} \}$. Let $P$ be the statement
$$\bigcup_{k \in I} [k,2k]= \mathbb{R}.$$

\begin{enumerate}[ (a)]    
	\item Expressing $P(x)$ using quantifiers, we have
	\begin{align*}
		\bigcup_{k \in I} [k,2k] & = \{x : \exists k \in I \text{ s.t. } x \in [k, 2k] \} = \mathbb{R} \\
		& = \{x  : \forall x \in \mathbb{R}, \; \exists k \in I \text{ s.t. } x \in [k, 2k] \} \\
		& = \{x : \forall x \in \mathbb{R}, \exists n \in \mathbb{Z} \text{ s.t. } x \in [n^2, 2n^2] \}
	\end{align*}
	
	\item $P(x)$: \textbf{False}
	
	\textbf{Proof:} Using Roster Notation, let us re-express the set $I$ as
	$$I = \{0,1,4,9,16,26, \ldots \}.$$
	Then the statement $P$ is given by
	$$\bigcup_{k \in I} [k,2k] = [0,0] \cup [1,2] \cup [4,8] \cup [9,18] \cup \cdots $$
	which is not equivalent to the set of real numbers $\mathbb{R}$. \QEDA
\end{enumerate}


\section*{Question 7}

Let $A=\{x \mid \forall n \geq 3, \; 1/n < x < 1 - 1/n \}.$

\begin{enumerate}[ (a)]    

    \item Let $I=\{n \in \mathbb{N} : n \geq 3 \}$ and $A_n = (1/n, \; 1-1/n)$. Then the set $A$ can be represented as an intersection of an indexed collection of sets:
    \begin{align*}
    	A & = \bigcap_{n = 3}^{\infty} \Big(\frac{1}{n}, \; 1- \frac{1}{n}\Big) \\
	& = \bigcap_{n \in I} A_n
    \end{align*}

    
    \item We can also express the set $A$ as an interval in $\mathbb{R}$: $A = (0, 1)$ or equivalently, $A= ] 0, 1 [$.
\end{enumerate}


\section*{Question 8}

For the two statements
\begin{itemize}
	\item Among the inhabitants of QE220 who can watch TV, not all have antennae on their head.
	\item The inhabitants of QE220 that are green and do not have antennae, cannot watch TV.
\end{itemize}

we will express them using quantifiers. For the universal set $\Omega$ of inhabitants of the planet QE220, let $T, A,$ and $G$ be subsets of $\Omega$ such that $T$ is the set that can watch TV, $A$ the set that have an antennae on their head, and $G$ the set that are green. Then we can express the two statements as

\begin{itemize}
	\item Statement 1: If $x \in T,$ then $\exists x \notin A$.
	\item Statement 2: If $y \in (G \cap \overline{A}),$ then $y \notin T$.
\end{itemize}

and the statement \say{not all the inhabitants of QE220 that can watch TV are green} that we want to determine the truth value of as
\begin{itemize}
	\item Statement 3: $\exists z \in T,$ such that $z \in G$.
\end{itemize}

Taking the contrapositive of Statement 2, we have that
\begin{align*}
	\text{If } y \in (G \cap \overline{A}), \text{ then } y \notin T & \equiv \text{If } y \in T, \text{then } y \notin (G \cap \overline{A})
	%%& \equiv \text{If } y \in T, \text{then } y \in (\overline{G} \cup A)
\end{align*}
Thus for both Statement 1 and Statement 2, if an element is in $T$ then there is no information as to whether it is also in the set $G$. Therefore the statement \say{not all the inhabitants of QE220 that can watch TV are green} does not follow from the given two statements. \QEDA










%This can be expressed as a Venn Diagram (Figure \ref{QE220}), where pink represents the known non-empty sets.
%\begin{figure}[h]
%\begin{center}
%\begin{tikzpicture}
%	\def\firstcircle{(0,0) circle (1)}		  				%Draw circle: (center x, center y) circle (radius) 
%         \def\secondcircle{(1,0) circle (1)}
%         \def\thirdcircle{(.5, 1) circle (1)}
%         
%          \fill[pink] \thirdcircle;
%          \fill[white] \secondcircle;
%
%           
%
%           
%            % outline & Labels
%            \draw \firstcircle 
%            	     (-1,0)  node [text=black,left] {$A$}    					%Label: (label x, label y) [specifiy colour, location]  {label text}
%                      \secondcircle (2,0)  node [text=black,right] {$G$}                           
%                      \thirdcircle (.5, 2) node [text=black,above] {$T$};
%%\caption{HIII}
%\end{tikzpicture}
%\end{center}
%\caption{Venn diagram of the sets $A, T,$ and $G$, where pink indicates a set that is known to be non empty.}
%\label{QE220}
%\end{figure}




%\begin{figure}[h]
%\begin{center}
%\begin{tikzpicture}
%            \def\firstcircle{(0,0) circle (1)}		  				%Draw circle: (center x, center y) circle (radius) 
%            \def\secondcircle{(1,0) circle (1)}
%            \def\thirdcircle{(.5, 1) circle (1)}
%            \def\boundingbox{(-2, -1.5) rectangle (3, 2.75)}		 %Draw rectangle: (LB x, LB y) rectangle (UB x, UB y); LB & UB on diagonal from each other
%           
%            % fill rectangle
%            \fill[pink] \boundingbox;
%	   
%	    %Intersect 1st & 3rd
%            \begin{scope}
%                \clip \boundingbox \thirdcircle;		%sample space
%                \clip \firstcircle;					%crop this part out
%                \fill[white] \secondcircle;			%fill this part in
%            \end{scope}
%            
%            %Intersect 2nd & 3rd
%            \begin{scope}
%                \clip \boundingbox \firstcircle;
%                \clip \thirdcircle;
%                \fill[white] \secondcircle;
%            \end{scope}
%            
%            %Intersect A \cap T^c
%            \begin{scope}
%                \clip \boundingbox \thirdcircle;
%                %\clip \firstcircle;
%                \fill[white] \firstcircle;				%fill this part in
%            \end{scope}
%
%           
%            % outline & Labels
%            \draw \firstcircle 
%            	     (-1,0)  node [text=black,left] {$A$}    					%Label: (label x, label y) [specifiy colour, location]  {label text}
%                      \secondcircle (2,0)  node [text=black,right] {$G$}                           
%                      \thirdcircle (.5, 2) node [text=black,above] {$T$}
%                      \boundingbox (3,2.8) node [text=black,above] {$U$}; 
%%\caption{HIII}
%\end{tikzpicture}
%\end{center}
%\caption{Venn diagram of the sets $A, T,$ and $G$, where pink indicates a set that is known to be non empty.}
%\label{QE220}
%\end{figure}




\section*{Question 9}

\textbf{Proposition:} If $x$ is an odd integer, then $9x+5$ is even.

\textbf{Proof:} Since $x$ is an odd integer, then there exists $k \in \mathbb{Z}$ such that
\begin{align*}
	x & = 2k+1 \\
	9x & = 18k+9 \\
	9x + 5 & = 18k + 9 + 5 \\
	& = 18k+14 \\
	& = 2(9k+7)
\end{align*}

Since $9k+7 \in \mathbb{Z}$, then $9x+5$ is even. \QEDA



\end{document} 