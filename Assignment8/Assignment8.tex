% ======================= Pre-Amble =========================
      
%Format
\documentclass[11pt, oneside]{article}   	% use "amsart" instead of "article" for AMSLaTeX format 
                     						%imports package {article} and specify option(s) [11pt, oneside]
\usepackage{geometry}                		% See geometry.pdf to learn the layout options. There are lots. 
    \geometry{letterpaper}                   		% ... or a4paper or a5paper or ... 
    %\geometry{landscape}                		% Activate for rotated page geometry

\usepackage[parfill]{parskip}    		        % Activate to begin paragraphs with an empty line rather than an indent

    %Colours
    \usepackage{graphicx, subcaption}
    \usepackage[usenames, dvipsnames]{color}     % font colour:    \textcolor{<colour>}{text}
          									%highlight text:  \colorbox{<color>}{text}
    \usepackage{soul}						%highlight text: \hl{}     %only  yellow								
    									%list of colours: https://www.sharelatex.com/learn/Using_colours_in_LaTeX
    									
    %Bullets
    \usepackage{enumerate}     %specify type of enumeration: \being{enumerate}[<type of enumeration>]
    
    %Footnote Spacing
    \setlength{\footnotesep}{0.4cm}                  %specify spacing b/w footnotes
    \setlength{\skip\footins}{0.6cm}                    % space b/w footnotes and textbody


%Mattematics
    %American Mathematics Society packages
    \usepackage{amsmath}	   %math
    \usepackage{amssymb}       %symbols
    \usepackage{amsthm}          %theorems

    %QED
    \newcommand*{\QEDA}{\hfill\ensuremath{\blacksquare}}         %make qed filled square:    \QEDA
    \newcommand*{\QEDB}{\hfill\ensuremath{\square}}               %make qed empty square: \QEDB 
    
    \renewcommand\qedsymbol{\ensuremath{\blacksquare}}		%Proof environment


%Figures
\usepackage{caption}
\captionsetup[figure]{labelfont=bf}    %make figure labels boldface
\captionsetup[table]{labelfont=bf}     %make table labels boldface

\usepackage[hidelinks]{hyperref}                % Allows for clickable references

    %Tables
    \usepackage[none]{hyphenat}                    % Stops breaking-up words in a table (i.e. no hyphens)                                                             
    
    \usepackage{array}   
    \newcolumntype{x}[1]{>{\centering\let\newline\\\arraybackslash\hspace{0pt}}p{#1}}       %center fixed column width: x{<len>}                      
    \newcolumntype{$}{>{\global\let\currentrowstyle\relax}}                                                   % let us apply things (e.g. bold/italicize) to entire row            
    \newcolumntype{^}{>{\currentrowstyle}}
    \newcommand{\rowstyle}[1]{\gdef\currentrowstyle{#1} #1\ignorespaces}
    
    %Images
    \graphicspath{ {images/} }                          %directory that your images are located in within your current directory
    
    %Diagrams
    \usepackage[latin1]{inputenc}
    \usepackage{tikz}
    \usepackage{tkz-berge}
    \usetikzlibrary{shapes,arrows}


%Bibliography
\usepackage[numbers,sort&compress]{natbib}   %for multiple references: sorts  (i.e. [1,2] NOT [2, 1] )
                                           				  %                                     compresses (i.e. [1-3] )
\usepackage[nottoc]{tocbibind}                            %add bibliography to table of contents


%Miscellaneous
\usepackage{dirtytalk}    %quotations: use \say  
\newcommand{\N}{\mathbb N}
\newcommand{\Z}{\mathbb Z}
\newcommand{\R}{\mathbb R}
\newcommand{\Q}{\mathbb Q}


%=========== Header & Footer =========================
\usepackage{fancyhdr}
\usepackage{lastpage}      %ensures you can reference LastPage (i.e. Page 2 of 10)

\renewcommand{\headrulewidth}{0.4pt}		%Decorative Header line: thickness={0.4pt}
\renewcommand{\footrulewidth}{0.4pt}		%Decorative Footer line: thickness={0.4pt}

\setlength{\headheight}{13.6pt} 		%space b/w top of page & header
\setlength{\headsep}{0.3in}		%space b/w page header and body

%Make Header & Footer    
\pagestyle{fancy}
    \lhead{Stephanie Knill} 		% controls the left corner of the header
    \chead{} 					% controls the center of the header
    \rhead{} 					% controls the right corner of the header
    \lfoot{} 					% controls the left corner of the footer
    \cfoot{Page~\thepage\ of \pageref{LastPage}} 				% controls the center of the footer
    												%Page~\thepage\  if just want Page x
    \rfoot{}			 		% controls the right corner of the footer

% ======================== Document ======================
\begin{document}

% Title Page
\title{MATH 220 --- Assignment 8 \\
\line(1,0){360} \\              %(slope x, y){length of line}
}
\author{
Stephanie Knill \\
54882113 \\
Due: March 17, 2016}

\date{}                   % Activate:  display a given date (e.g. {August 4} ) or no date (empty {} )
                                    %No activate: display current date
\maketitle

%\thispagestyle{empty}                   %Remove header from this (first) page. Change empty -> plain to keep numbering
%								-> Doesn't matter in this case (b/c title page)
%\cleardoublepage


% ================= Questions ================

\section*{Question 1}

Find the domain and range of the function $f(x) = \frac{\sqrt{x-1}}{x}$ (assume $x$ is real).

\emph{Domain}

For the numerator we have that $x-1 \geq 0$, so $x\geq 1$. In the denominator $x \neq 0$. Thus overall we have
$$dom(f) = \{x \in \mathbb{R} \mid x \geq 1\}$$

\begin{proof}
Assume to the contrary that $x \notin dom(f)$. Let $x = 1 - \delta$, where $\delta$ non-negative constant. Substituting this into $f(x)$ we have
$$f(x) = \frac{\sqrt{(x-\delta)-1}}{x}$$
which is undefined. Thus we have arrived at a contradiction and $x \in dom(f)= \{x \in \mathbb{R} \mid x \geq 1\}$.
\end{proof}

\emph{Range}

Since $x \geq 1$, we have the minimal value of $f(x)$ to be at $f(1) = 0$ and the maximal value of $f(x)$ to be at $f(1) = 1/2$. Thus we have that
$$range(f) = \{f(x) \in \mathbb{R} \mid 0 \leq y \leq 1/2 \}$$

\begin{proof}
Assume to the contrary that $f(x) \notin range(f)$. The we have that $f(x) = 0 - \delta$ or $f(x) = 1/2 + \alpha$, where $\delta, \alpha$ non-negative constant. Substituting in we have
$$0 - \delta = \frac{\sqrt{x-1}}{x}$$
$$\delta = - \frac{\sqrt{x-1}}{x}$$
which is a contradiction since $\delta > 0$. Similarly in the case of $f(x) = 1/2 + \alpha$ we also derive a contradiction. Thus we have that $range(f) = \{f(x) \in \mathbb{R} \mid 0 \leq y \leq 1/2 \}$.
\end{proof}

\section*{Question 2}

Let $f : \mathbb{R} \to \mathbb{R}$ be the function defined by $f(x)=x^2+3x+4$.
\begin{enumerate}[\quad(a)]
	\item $f$ is not injective.
		\begin{proof} 
		Let $x_1=0$ and $x_2=-3$. Then $f(x_1) = 4 = f(x_2)$, thus $f$ is not injective.
		\end{proof}
	\item Find all pairs $a,b$ of real numbers so that $f(a) = f(b)$.
	
	Let $f(a) = f(b)$, where $a,b \in \mathbb{R}$. Then
	\begin{align*}
		a^2+3a+4 & = b^2+3b+4 \\
		a^2+3a-b^2-3b & = 0 \\
		(a^2-b^2) + 3(a-b) & = 0 \\
		(a-b)(a+b+3) & = 0
	\end{align*}
	Thus for $f(a)=f(b)$, $a=b$ or $a=-(b+3)$. In other words, we have that $f$ is injective when $$\{(a,b) \in \mathbb{R} \mid a=b \text{ or } a=-(b+3) \}$$
\end{enumerate}


\section*{Question 3}

 Let $h:\mathbb{Z} \to \mathbb{Z}$ be a function defined by $h(n) = 3n-8$.
\begin{enumerate}[\quad(a)]
	\item Prove: $f$ is injective.
		\begin{proof} 
		Assume $h(a) = h(b)$, where $a,b \in \mathbb{Z}$. Then
		\begin{align*}
			3a-8 & = 3b-8 \\
			3a & = 3b \\
			a& = b 
		\end{align*}
		Thus $f$ is injective.
		\end{proof}
	\item Disprove: $f$ is surjective.
	
	Let $h(n) =0 \in \mathbb{Z}$. Then
	\begin{align*}
		3n-8 & = 0 \\
		n & = \frac{8}{3} 
	\end{align*}
	Since $n=\frac{8}{3} \notin \mathbb{Z}$, $f$ is not surjective.
\end{enumerate}

\section*{Question 4}

Give an example of a function $f : \N \to \N$ that is
\begin{enumerate}[\quad(a)]
	\item one-to-one and onto
	$$f(x) = x$$
	\item one-to-one but not onto
	$$f(x)=x+1$$.
	Here, range($f)=\N-\{1\}.$
	\item onto but not one-to-one
	\begin{align*}
		f(x) = \begin{cases}
				x-1 & \text{if } x \geq 2 \\
				x & \text{if } x = 1
			\end{cases}
	\end{align*}
	\item neither one-to-one or onto
	\begin{align*}
		f(x) = \begin{cases}
				1 & \text{if } x \text{ odd} \\
				2 & \text{if } x \text{ even}
			\end{cases}
	\end{align*}
\end{enumerate}


\section*{Question 5}

Prove or disprove: for every set $A$ there is an injective function $f : A \to \mathcal{P}(A)$
\begin{proof}\footnote{A counterexample may have also sufficed. Let $A=\{1,2\}$, then $\mathcal{P}(A) = \{\emptyset, \{1\}, \{2\}, \{1,2\}\}$. Thus $|A| = 2 < 4 = \mathcal{P}(A)$ and $f$ is not injective. \QEDA}
For a function $f : A \to \mathcal{P}(A)$ to be injective, then
$$|A| \geq |\mathcal{P}(A)|$$
However we know that $\mathcal{P}(A) = 2^{|A|}$. Thus $|A| \geq 2^{|A|}$ and the function $f$ is not injective.
\end{proof}



\section*{Question 6}

Give an example of
\begin{enumerate}[\quad(a)]
	\item A bijection from (0,1) to $(1,\infty)$
	$$f(x) = tan\Big(\frac{\pi x}{2}\Big)+1$$
	\begin{proof}
	Let us first prove that $f$ is injective. Assume that $f(a) = f(b)$. Then
	\begin{align*}
		tan\Big(\frac{\pi a}{2}\Big)+1 & = tan\Big(\frac{\pi b}{2}\Big)+1 \\
		tan\Big(\frac{\pi a}{2}\Big) & = tan\Big(\frac{\pi b}{2}\Big) 
	\end{align*}
	Since $a, b \in (0,1)$ and the tangent has a period of $\pi$, then for $tan\Big(\frac{\pi a}{2}\Big) = tan\Big(\frac{\pi b}{2}\Big)$, we must have that $a=b$. Thus $f$ is injective. Let us now prove that $f$ is also surjective. Let $r \in (1,\infty)$ and $x = \frac{2}{\pi} tan^{-1}(r-1)$. Then
	\begin{align*}
		f(x) = tan \Big( \frac{\pi}{2} \cdot \frac{2}{\pi} tan^{-1}(r-1)\Big) +1 = (r-1) + 1 = r
	\end{align*}
	thereby making $f$ also surjective. Since $f$ is both injective and surjective, we can conclude that $f$ is bijective.
	\end{proof}
	
	\item A surjection from $\R$ to $\mathbb{Q}$
	\begin{align*}
		f(x) = \begin{cases}
				x & \text{if } x \in \mathbb{Q} \\
				0 & \text{if } x \notin \mathbb{Q}
			\end{cases}
	\end{align*}
	\begin{proof}
	Let $r \in \Q$. We will us examine the case where $x \in \Q$. Here
	$$f(x) = f(r) = r$$
	Which means we hit every point in the range, thus making it surjective. \footnote{Although we do not need to examine the case where $x \notin \Q$, we have $f(x) = f(0) = 0 \in \Q$ which makes this function only surjective, rather than bijective.}
	\end{proof}
	
	\item A bijection from $\N$ to $\Z$
	\begin{align*}
		f(x) = \begin{cases}
				-\frac{x}{2} & \text{if } x \text{ even} \\
				\frac{x-1}{2} & \text{if } x \text{ odd}
			\end{cases}
	\end{align*}
	\begin{proof}
	Let us first prove that $f$ is injective. Assume that $f(a) = f(b)$. In the case where $x$ is even
	\begin{align*}
		-\frac{a}{2} & = -\frac{b}{2} \\
		a& = b
	\end{align*}
	Similarly in the case where $x$ is odd
	\begin{align*}
		\frac{a-1}{2} & = \frac{b-1}{2} \\
		a& = b
	\end{align*}
	Thus $f$ is injective. Let us now prove that $f$ is also surjective. Let $r \in \Z$. In the first case where $x$ is even, let $x=-2r$. Then
	$$f(-2r)=-\frac{-2r}{2} = r$$
	Similarly in the case where $x$ is odd, let $x=2r+1$. Then
	$$f(2r+1) = \frac{(2r+1)-1}{2} = r$$
	thereby making $f$ also surjective. Since $f$ is both injective and surjective, we can conclude that $f$ is bijective.
	\end{proof}
\end{enumerate}



\section*{Question 7}

Let $f:A \to B$ be a function and if $D\subset B$, recall that the inverse image of $D$ under $f$ is by definition the set 
$f^{-1}(D)=\{x\in A:f(x)\in D\}$.  
Note that $f^{-1}(D)$  is a \emph{set}, and is defined for any function $f$, even if $f$ does not have an inverse.

\begin{enumerate}[\quad(a)]
	\item If $D_1, D_2 \subseteq B$, prove that 
	\begin{align*}
 		 f^{-1}(D_1 \cap D_2) = f^{-1}(D_1) \cap f^{-1}(D_2)  
	\end{align*}
	\begin{proof}
		$\subseteq$ Let $x \in f^{-1}(D_1 \cap D_2)$. So $f(x) \in D_1 \cap D_2$. Since $f(x) \in D_1$ and $f(x) \in D_2$, we have that $x \in f^{-1}(D_1)$ and $x \in f^{-1}(D_2)$. Thus $x \in f^{-1}(D_1) \cap f^{-1}(D_2)$ and we have that $f^{-1}(D_1 \cap D_2) \subseteq f^{-1}(D_1) \cap f^{-1}(D_2)$ 
		
		$\supseteq$ Let $x \in f^{-1}(D_1) \cap f^{-1}(D_2)$. Then $x \in f^{-1}(D_1)$ and $x \in f^{-1}(D_2)$, so we have that $f(x) \in D_1$ and $f(x) \in D_2$. Since $f(x) \in D_1 \cap D_2$, then $x \in f^{-1}(D_1 \cap D_2)$. Thus we can conclude that $f^{-1}(D_1 \cap D_2) \supseteq f^{-1}(D_1) \cap f^{-1}(D_2)$ 
		
	\end{proof}
	
	\item If $D_1, D_2 \subseteq B$, prove that 
	\begin{align*}
  		f^{-1}(D_1 \cup D_2) = f^{-1}(D_1) \cup f^{-1}(D_2)  
	\end{align*}
	\begin{proof}
		$\subseteq$ Let $x \in f^{-1}(D_1 \cup D_2)$. So $f(x) \in D_1 \cup D_2$. Since $f(x) \in D_1$ or $f(x) \in D_2$, we have that $x \in f^{-1}(D_1)$ or $x \in f^{-1}(D_2)$. Thus $x \in f^{-1}(D_1) \cup f^{-1}(D_2)$ and we have that $f^{-1}(D_1 \cup D_2) \subseteq f^{-1}(D_1) \cup f^{-1}(D_2)$ 
		
		$\supseteq$ Let $x \in f^{-1}(D_1) \cup f^{-1}(D_2)$. Then $x \in f^{-1}(D_1)$ or $x \in f^{-1}(D_2)$, so we have that $f(x) \in D_1$ or $f(x) \in D_2$. Since $f(x) \in D_1 \cup D_2$, then $x \in f^{-1}(D_1 \cup D_2)$. Thus we can conclude that $f^{-1}(D_1 \cup D_2) \supseteq f^{-1}(D_1) \cup f^{-1}(D_2)$ 

	\end{proof}
	
	\item If $D\subset B$, prove that
	\begin{align*}
  		f^{-1}(B - D) = A - f^{-1}(D)
	\end{align*}
	\begin{proof}
	$\subseteq$ Let $x \in f^{-1}(B-D)$. Then
	\begin{align*}
		x & \in f^{-1}(B \cap D^c) \\
		x & \in f^{-1}(B) \cap f^{-1} (D^c) \qquad \text{(by proof 7b)}\\
		x & \in A \cap f^{-1} (D^c) \\
		x & \in A - f^{-1} (D)
	\end{align*}
	$\supseteq$ Let $x \in A - f^{-1} (D)$. Then
	\begin{align*}
		x & \in A \cap f^{-1} (D^c) \\
		x & \in f^{-1}(B) \cap f^{-1} (D^c) \\
		x & \in f^{-1}(B \cap D^c) \\
		x & \in f^{-1}(B-D)
	\end{align*}
	\end{proof}
\end{enumerate}

\section*{Question 8}

Let $A$ be a well-ordered set. Let $f:{\mathcal P}(A)\to A$ be the function that assigns to every  $B\subseteq A$ the smallest element of $B$. 
\begin{enumerate}[\quad (a)]
	\item Is $f$ a function?
	\begin{proof}
	Let $x \in B$. Since $B \subseteq A$ and $A$ is a well-ordered set, then $B$ contains a minimal element. Thus for every input $B$ we have a minimal element (we need not worry about the case where $B$ is the empty set), thus a single output. By definition then $f$ is a function. 	\end{proof}
	\item Is $f$ injective?
	\begin{proof}
		Let $A=\{1,2\}$. Then for $B=\{1\}$ and $B=\{1,2\}$, both have a minimal element of 1, thereby making $f$ not injectvie.
	\end{proof}
	\item Is $f$ surjective?
	\begin{proof}
		Since $A$ is a well-ordered set, then every subset of $A$, in other words the power set of $A$, contains a minimal element. Thus each subset within the power set $\mathcal{P}(A)$ will map onto an element in $A$. Since a subset of cardinality 1 will always have a minimal element of itself, we know that $\mathcal{P}(A)$ has the additional property of mapping onto all elements of $A$, thereby making $f$ surjective.
	\end{proof}
\end{enumerate}

\section*{Question 9}
\begin{enumerate}[(a)]
	\item Prove that if $f:A\to B$ is an injective function, then $f(C_1\cap C_2)=f(C_1)\cap f(C_2)$ for all $C_1, C_2\subseteq A$.
	\begin{proof}
		$\subseteq$ Let $x \in f(C_1\cap C_2)$. Then
		$$f^{-1}(x) \in C_1 \cap C_2$$
		$$f^{-1}(x) \in C_1 \text{ and } f^{-1}(x) \in C_2$$
		$$x \in f(C_1) \text{ and } x \in f(C_2)$$
		$$x \in f(C_1) \cap f(C_2)$$
%		\begin{align*}
%			f^{-1}(x) \in C_1 \cap C_2 \\
%			f^{-1}(x) \in C_1 \text{ and } f^{-1}(x) \in C_2 \\
%			x \in f(C_1) \text{ and } x \in f(C_2) \\
%			x \in f(C_1) \cap f(C_2)
%		\end{align*}
		Thus $f(C_1\cap C_2) \subseteq f(C_1)\cap f(C_2)$.
		
		$\supseteq$ Let $x \in  f(C_1) \cap f(C_2)$ Then $x \in f(C_1)$ and $x \in f(C_2)$. Since $x \in f(C_1)$, $\exists y_1 \in A$ such that $f(y_1)=x$. Similarly, since $x \in f(C_2)$, $\exists y_2 \in A$ such that $f(y_2)=x$. Since $f(y_1) = x = f(y_2$ and $f$ is an injection, then $y_1=y_2$. Therefore $y_1 \in C_2$ and $y_1 \in C_1 \cap C_2$. Thus we have that $x=f(y_1) \in f(C_1 \cap C_2)$ and we have shown that $f(C_1\cap C_2) \supseteq f(C_1)\cap f(C_2)$.
		
	\end{proof}
	
	\item Prove that if $f:A\to B$ is a surjective function, then $f(f^{-1}(D))=D$ for every subset $D\subseteq B$.
	\begin{proof}
	$\subseteq$ Let $x \in f(f^{-1}(D))$. Then $\exists a \in f^{-1}(D)$ such that $f(a)=x$. By the definition of pre-image, $x \in D$ and we are done.
	
	$\supseteq$ Let $x \in D$. Since $f$ is a surjection, there exists $a \in A$ such that $f(a)=x$. Thus $a \in f^{-1}(D)$ and we have that $f(a) \in f(f^{-1}(D))$. Here $x=f(a)$, so $x \in f(f^{-1}(D))$.
	
	\end{proof}
\end{enumerate}




















\end{document} 