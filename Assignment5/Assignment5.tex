% ======================= Pre-Amble =========================
      
%Format
\documentclass[11pt, oneside]{article}   	% use "amsart" instead of "article" for AMSLaTeX format 
                     						%imports package {article} and specify option(s) [11pt, oneside]
\usepackage{geometry}                		% See geometry.pdf to learn the layout options. There are lots. 
    \geometry{letterpaper}                   		% ... or a4paper or a5paper or ... 
    %\geometry{landscape}                		% Activate for rotated page geometry

\usepackage[parfill]{parskip}    		        % Activate to begin paragraphs with an empty line rather than an indent

    %Colours
    \usepackage{graphicx, subcaption}
    \usepackage[usenames, dvipsnames]{color}     % font colour:    \textcolor{<colour>}{text}
          									%highlight text:  \colorbox{<color>}{text}
    \usepackage{soul}						%highlight text: \hl{}     %only  yellow								
    									%list of colours: https://www.sharelatex.com/learn/Using_colours_in_LaTeX
    									
    %Bullets
    \usepackage{enumerate}     %specify type of enumeration: \being{enumerate}[<type of enumeration>]
    
    %Footnote Spacing
    \setlength{\footnotesep}{0.4cm}                  %specify spacing b/w footnotes
    \setlength{\skip\footins}{0.6cm}                    % space b/w footnotes and textbody


%Mattematics
    %American Mathematics Society packages
    \usepackage{amsmath}	   %math
    \usepackage{amssymb}       %symbols
    \usepackage{amsthm}          %theorems

    %QED
    \newcommand*{\QEDA}{\hfill\ensuremath{\blacksquare}}         %make qed filled square:    \QEDA
    \newcommand*{\QEDB}{\hfill\ensuremath{\square}}               %make qed empty square: \QEDB 
    
    \renewcommand\qedsymbol{\ensuremath{\blacksquare}}		%Proof environment


%Figures
\usepackage{caption}
\captionsetup[figure]{labelfont=bf}    %make figure labels boldface
\captionsetup[table]{labelfont=bf}     %make table labels boldface

\usepackage[hidelinks]{hyperref}                % Allows for clickable references

    %Tables
    \usepackage[none]{hyphenat}                    % Stops breaking-up words in a table (i.e. no hyphens)                                                             
    
    \usepackage{array}   
    \newcolumntype{x}[1]{>{\centering\let\newline\\\arraybackslash\hspace{0pt}}p{#1}}       %center fixed column width: x{<len>}                      
    \newcolumntype{$}{>{\global\let\currentrowstyle\relax}}                                                   % let us apply things (e.g. bold/italicize) to entire row            
    \newcolumntype{^}{>{\currentrowstyle}}
    \newcommand{\rowstyle}[1]{\gdef\currentrowstyle{#1} #1\ignorespaces}
    
    %Images
    \graphicspath{ {images/} }                          %directory that your images are located in within your current directory
    
    %Diagrams
    \usepackage[latin1]{inputenc}
    \usepackage{tikz}
    \usepackage{tkz-berge}
    \usetikzlibrary{shapes,arrows}


%Bibliography
\usepackage[numbers,sort&compress]{natbib}   %for multiple references: sorts  (i.e. [1,2] NOT [2, 1] )
                                           				  %                                     compresses (i.e. [1-3] )
\usepackage[nottoc]{tocbibind}                            %add bibliography to table of contents


%Miscellaneous
\usepackage{dirtytalk}    %quotations: use \say  



%=========== Header & Footer =========================
\usepackage{fancyhdr}
\usepackage{lastpage}      %ensures you can reference LastPage (i.e. Page 2 of 10)

\renewcommand{\headrulewidth}{0.4pt}		%Decorative Header line: thickness={0.4pt}
\renewcommand{\footrulewidth}{0.4pt}		%Decorative Footer line: thickness={0.4pt}

\setlength{\headheight}{13.6pt} 		%space b/w top of page & header
\setlength{\headsep}{0.3in}		%space b/w page header and body

%Make Header & Footer    
\pagestyle{fancy}
    \lhead{Stephanie Knill} 		% controls the left corner of the header
    \chead{} 					% controls the center of the header
    \rhead{} 					% controls the right corner of the header
    \lfoot{} 					% controls the left corner of the footer
    \cfoot{Page~\thepage\ of \pageref{LastPage}} 				% controls the center of the footer
    												%Page~\thepage\  if just want Page x
    \rfoot{}			 		% controls the right corner of the footer

% ======================== Document ======================
\begin{document}

% Title Page
\title{MATH 220 --- Assignment 5 \\
\line(1,0){360} \\              %(slope x, y){length of line}
}
\author{
Stephanie Knill \\
54882113 \\
Due: February 9, 2016}

\date{}                   % Activate:  display a given date (e.g. {August 4} ) or no date (empty {} )
                                    %No activate: display current date
\maketitle

%\thispagestyle{empty}                   %Remove header from this (first) page. Change empty -> plain to keep numbering
%								-> Doesn't matter in this case (b/c title page)
%\cleardoublepage


% ================= Questions ================

\section*{Question 1}

\textbf{Proposition:} $5x-11$ is even if and only if $x$ is odd.


\emph{Proof}

$\Rightarrow$ We will express the forward direction in terms of its contrapositive \say{If $x$ is even, then $5x-11$ is odd} and prove this logically equivalent statement. Let there exist an integer $k$ such that $x=2k$. Then
\begin{align*}
	x & =2k \\
	5x & = 10k \\
	5x -11 & = 10k - 11 \\
	& = 10k - 12 + 1 \\
	& = 2(5k-6) + 1 \\
\end{align*}
Since $5k-6$ is an integer, then $5x-11$ is odd and we have proved the forward direction.

$\Leftarrow$ Assume that $x$ is odd. Then there exists an integer $q$ such that $x=2q+1$ and we have that
\begin{align*}
	x & =2q+1 \\
	5x & = 10q + 5 \\
	5x -11 & = 10q -6 \\
	& = 2(5q-3)
\end{align*}
Since $5q-3$ is an integer, then $5x-11$ is even. \QEDA

\section*{Question 2}

\textbf{Proposition:} the product of two integers $ab$ is odd if and only if both $a$ and $b$ are odd.

\begin{proof}
$\Rightarrow$ We will express the forward direction in terms of its contrapositive \say{If $a$ and $b$ are both even, then $ab$ is even} and prove this logically equivalent statement. Let there exist integers $k_1$ and $k_2$ such that $a=2k_1$ and $b=2k_2$. Then the product of these two is given by
\begin{align*}
    	ab & = (2k_1)(2k_2) \\
    	& = 4k_1k_2 \\
    	& = 2(2k_1k_2)
\end{align*}
Since $2k_1k_2$ is an integer, then $ab$ is even and we have proved the forward direction.

$\Leftarrow$ Assume that $a$ and $b$ are both odd. Then there exists integers $q_1$ and $q_2$ such that $a=2q_1+1$ and $b=2q_2+1$. The product of these two integers can be computed as
\begin{align*}
    	ab & = (2q_1+1)(2q_2+1) \\
    	& = 4q_1q_2 + 2q_1 + 2q_2 + 1 \\
    	& = 2(2q_1q_2 + q_1 + q_2) + 1
\end{align*}
Since $2q_1q_2 + q_1 + q_2$ is an integer, then $ab$ is odd.
\end{proof}

\section*{Question 3}

\textbf{Proposition:} If $n$ is even, then $n^3$ is even.

\begin{proof}
Let there exist a $k \in \mathbb{Z}$ such that $n=2k$. Then
\begin{align*}
	n^3 & = (2k)^3 \\
	& = 8k^3 \\
	& = 2(4k^3)
\end{align*}
Since $4k^3 \in \mathbb{Z}$, then $n^3$ is even.
\end{proof}


\section*{Question 4}

\textbf{Proposition:} For any sets $A$ and $B$, $A\Delta B = \emptyset$ iff $A=B$.

\begin{proof}
$\Rightarrow$ We will express the forward direction in terms of its contrapositive \say{If $A \neq B$, then $A \Delta B \neq \emptyset$} and prove this logically equivalent statement. By definition of the symmetric set difference, we have that
$$A \Delta B = (A - B) \cup (B - A)$$

Since $A \neq B$, then both $(A - B)$ and $(B-A)$ do not equal the empty set. By definition, the union of two nonempty sets is not the empty set. Therefore it follows that $A \Delta B = (A - B) \cup (B - A) \neq \emptyset$

$\Leftarrow$ Assume that $A = B$. Then
\begin{align*}
	A \Delta B & = A \Delta A \\
	& = (A-A) \cup (A-A) \\
	& = (A-A) \\
	& = A \cap \overline{A} \\
	& = \emptyset
\end{align*}
\end{proof}

\section*{Question 5}

\textbf{Conjecture:} For any sets $A$ and $B$, $(A \cup B) - B = A$

\textbf{Counterexample}
Let $A = \{1,2,3\}$ and $B=\{2,3,4\}$. Then $(A \cup B) = \{1,2,3,4\}$ and we have that $(A \cup B) - B = \{1\}$. However, this does not equal the set $A$. Thus the above conjecture is False.

\section*{Question 6}

\textbf{Proposition:} For any sets $A, B,$ and $C$, $(A-B) \cup (A-C) = A - (B \cap C)$.

\begin{proof}
Using our fundamental properties of set operations, we have 
\begin{align*}
	(A-B) \cup (A-C) & = (A \cap B^c) \cup (A \cap C^c) \quad \textit{(definition of set difference)}\\
	& = A \cap (B^c \cup C^c) \quad \textit{(distributive laws)} \\
	& = A \cap (B \cap C)^c \quad \textit{(De Morgan's laws)}\\
	& = A - (B \cap C) \quad \textit{(definition of set difference)}
\end{align*}
\end{proof}

\section*{Question 7}

\textbf{Proposition:} For sets $A, B, C,$ and $D$, $(A \times B) \cap (C \times D) = (A \cap C) \times (B \cap D)$.

\begin{proof}
We will first show that $(A \times B) \cap (C \times D) \subseteq (A \cap C) \times (B \cap D)$. Let $(x,y) \in (A \times B) \cap (C \times D)$. Then $x \in A$ and $x \in C$ and $y \in B$ and $y \in D$. Therefore $x \in (A \cap C)$ and $y \in (B \cap D)$. Combining these, we have that
$$(x,y) \in (A \cap C) \times (B \cap D)$$
implying that $(A \times B) \cap (C \times D) \subseteq (A \cap C) \times (B \cap D)$.

Let us now show that $(A \times B) \cap (C \times D) \supseteq (A \cap C) \times (B \cap D)$. Let $(u,v) \in (A \cap C) \times (B \cap D)$. Then $u \in A$ and $u \in C$ and $v \in B$ and $v \in D$. Rearranging, we have that $(u,v) \in (A \times B)$ and $(u.v) \in (C \times D)$. Combining these gives us
$$(u,v) \in (A \times B) \cap (C \times D)$$
which completes our proof of $(A \times B) \cap (C \times D) \supseteq (A \cap C) \times (B \cap D)$, thereby proving set equality.
\end{proof}


\section*{Question 8}

\textbf{Proposition:} Let $a, b \in \mathbb{Z}, a,b \neq 0$. If $a \mid b$ and $b \mid a$, then $a=b$ or $a=-b$.

\begin{proof}
Assume that $a \mid b$ and $b \mid a$. Then there exists integers $k_1$ and $k_2$ such that $a=bk_1$ and $b=ak_2$. Substituting in we have that
\begin{align*}
	a & = bk_1 \\
	& = (ak_2)k_1 \\
	k_1k_2 & = 1 \\
\end{align*}
Since $k_1, k_2 \in \mathbb{Z}$, then $k_1$ and $k_2$ must both equal either 1 or -1. In the first case where $k_1=k_2=1$, we have that $a=b$ and $b=a$. In the second case where $k_1 = k_2 = -1$, $a=-b$ and $b=-a$. Therefore if  $a \mid b$ and $b \mid a$, we have that either $a=b$ or $a=-b$.
\end{proof}

\section*{Question 9}

\textbf{Proposition:} Let $x,y \in \mathbb{Z}$. If $3 \nmid x$ and $3 \nmid y$, then $3 \mid (x^2 - y^2)$.

\begin{proof}
Assume that $3 \nmid x$ and $3 \nmid y$. Then there exists remainders $r_1, r_2 \in \{1,2\}$ such that $x \equiv r_1$ mod 3 and $y \equiv r_2$ mod 3. By the Properties of Congruence Theorem, we have
\begin{align*}
	x^2-y^2 & \equiv (r_1 \cdot r_1) - (r_2 \cdot r_2) \text{ mod } 3 \\
	& \equiv r_1^2 - r_2^2 \text{ mod } 3
\end{align*}
Since $r_1, r_2 \in \{1,2\}$, we can break this into 4 cases:

\textbf{Case 1:} $r_1=r_2=1$. Here we have that
\begin{align*}
	x^2-y^2 & \equiv (1-1) \text{ mod } 3 \\
	& \equiv 0 \text{ mod } 3
\end{align*}
\textbf{Case 2:} $r_1=r_2=2$
\begin{align*}
	x^2-y^2 & \equiv (4-4) \text{ mod } 3 \\
	& \equiv 0 \text{ mod } 3
\end{align*}
\textbf{Case 3:} $r_1=1$ and $r_2=2$
\begin{align*}
	x^2-y^2 & \equiv (1-4) \text{ mod } 3 \\
	& \equiv -3 \text{ mod } 3 \\
	& \equiv 0 \text{ mod } 3
\end{align*}
\textbf{Case 4\footnote{Although this case is equivalent to Case 3 and therefore can be omitted, we have included it for completeness.}:} $r_1=2$ and $r_2=1$
\begin{align*}
	x^2-y^2 & \equiv (4-1) \text{ mod } 3 \\
	& \equiv 3 \text{ mod } 3 \\
	& \equiv 0 \text{ mod } 3
\end{align*}
Since in each case of values for $r_1$ and $r_2$, we have that $x^2-y^2 \equiv 0 \text{ mod } 3 $, we can conclude that $3 \mid (x^2-y^2)$.
\end{proof}


\end{document} 